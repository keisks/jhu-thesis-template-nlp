\chapter{Introduction}
\label{chap:intro}
\chaptermark{Optional running chapter heading}

This is a language game \cite{wittgenstein1953philosophical}.
By the way, \newcite{quine1951main} wrote ``Two dogmas of empiricism'',
\begin{displayquote}
  Modern empiricism has been conditioned in large part by two
  dogmas. One is a belief in some fundamental cleavage between
  truths which are analytic, or grounded in meanings independently of
  matters of fact, and truth which are synthetic, or grounded in fact.
  The other dogma is reductionism: the belief that each meaningful
  statement is equivalent to some logical construct upon terms which
  refer to immediate experience.
\end{displayquote}


A table is shown in Table \ref{tab:sample}.
\blindtext[3]

\begin{table}[t]
  \small
  \centering
  \begin{tabular}{l|c|c|c} \hline
       & A     & B & C \\ \hline\hline
  cat  & meow  &   &   \\
  dog  & woof  &   &   \\
  cow  & moo   &   &   \\ \hline
  \end{tabular}
  \caption{Sample table}
  \label{tab:sample}
  %\vspace{-2mm}
\end{table}

\blindtext[2]
A figure sample is shown in Figure \ref{fig:sample}.

\begin{figure}[t]
  \centering
  \includegraphics[width=70mm]{fig/sample_figure.pdf}
  \caption{Sample figure}
  \label{fig:sample}
  %\vspace{-2mm}
\end{figure}


\blindtext[3]

